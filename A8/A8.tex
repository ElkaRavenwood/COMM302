\documentclass[letterpaper]{article}
\usepackage{tikz}
\usepackage{float}
\usepackage{multicol}
\usepackage[margin=1in]{geometry}
\usepackage{hyperref}
\usepackage{placeins}
\usepackage[group-separator={,}]{siunitx}

\usepackage[
backend=biber,
style=apa,
citestyle=authoryear,
sorting=nty
]{biblatex}
\addbibresource{bib.bib}

\title{
    COMM 302 - W2021 - Assignment \#8 \\
    \begin{large}
        One Million Teachers - Venture Assessment
    \end{large}
}
\author{
    Jamie Won | 20113217 | \href{mailto:jamie.won@queensu.ca}{jamie.won@queensu.ca}
}

\begin{document}

\maketitle
\tableofcontents
\listoffigures

\cleardoublepage
\section{Executive Summary}

1 Million Teachers (1MT) is a for-profit organization that seeks to address the social stigma of an undesirable teaching career through the empowerment of teachers. It is the hope that empowered teachers will attract more students to the education industry, and when the students become successful teachers, will help offset the projected seven million teacher deficit in 2030 (H. Subair, personal communication, March 23, 2021). The company offers teachers training through their online portal in fields such as product management, inclusive teaching, entrepreneurship, critical and creative thinking, curriculum design and assessment guidelines(\cite{companysite-programs}) for a negotiable fee to private school owners. 1MT's biggest issue at the moment is acquiring new customers; the infrastructure and resources are available, there is merely not enough customers. It is recommended that 1MT become a non-profit, social-good enterprise and partner with other like-minded organizations. Through the use of resources from such organizations as well as the infrastructure 1MT has, it should be able to successfully combat the pervasive social stigma.

\section{Context}

    \subsection{Industry Analysis}
        
        \subsubsection{Company Overview}
            
            By 2030, the lack of teachers will be catastrophic for the futures of innumerable children. In Sub-Saharan Africa alone, it is projected that there will be a 17 million deficit in primary and secondary school teachers(\cite{companysite-about}). One Million Teachers (1MT for short) is a social enterprise who's goal is to empower teachers. Empowerment is defined as giving teachers the ability to connect with students, know they make a difference, "have a say" and see themselves as a part of their educational institution's leadership. By empowering teachers, 1MT hopes to inspire students to become [empowered] teachers, which, in turn, will change the stigma that teaching is a lesser profession.
            %  A detailed description of the product offered by 1MT can be found in Section \ref{Product}.

        \subsubsection{Industry Definition}
            
            IMT's industry is that of e-learning. A more formal name for this is Online Tutoring Services. The industry is defined as follows: "This industry provides tutoring services via the internet. Industry operators provide tutoring servies across a variety of academic subjects and all age groups" (\cite{industrydef}). 1MT offers resources primarily for product management, inclusive teaching, entrepreneurship, critical and creative thinking, curriculum design and assessment guidelines(\cite{companysite-programs}).

        \subsubsection{Relevant Trends}

            1MT targets the Sub-Saharan educational system because it is the place with the most to lose from a teacher deficit (\cite{companysite-about}). The trends discussed in this section pertain to this region. It should be noted that the strength of each trend also vastly differs throughout the region, with a stronger variance for the legal and political trends.

            While there are many social trends in the region, there are none relevant to the industry. 1MT aspires to create a trend towards an increased [empowered] teacher population. 1MT will be able to exploit the trend towards an environmentally-conscious world as it is an e-learning tool and physical materials can be saved. 1MT will need to ensure they operate legally in the region, but there are no relevant legal trends. The applicable political and technological trends are summarized in the table below. 
            
            \begin{figure}[h!]
                \begin{center}
                    \begin{tabular}{| p{7.5cm} | p{7.5cm} |}
                        \hline
                        Political & Technological\\
                        \hline
                        Increased number of government assistance programs &
                            Increased technology availability and use \\
                        Slowly decreasing corruption & 
                            Increased wifi and internet availability \\
                        \raggedright Very slow push towards equal representation (\cite{trends-politics-women}) & 
                            [For those who can afford it], increased e-learning tool use \\
                        A push towards bettering their country & Increased dependence on technology \\
                        \hline
                    \end{tabular} \\
                    \caption{Table depicting the political and technological trends}
                \end{center}
            \end{figure}

            Economically, COVID-19 has resulted in Sub-Saharan Africa's first recession in 25 years (\cite{trends-economical-covid}), ruining the trend towards poverty reduction. Research shows that "Those who had access to digital connectivity managed to cope better with isolation; the use of that technology permeated the way people interacted, helping to enhance use of social capital (mutual help), which is pivotal to African livelihoods" (\cite{trends-economical-covid}), implying an increased reliance on technology. This, in conjunction with the trends in the table above suggest that e-learning is a growing industry.

    \subsection{Segmentation and Targeting}

        \subsubsection{Segmentation}

            The market is split into two segments: public and private education systems. Although the public sector is much larger, it is harder to change and the system is corrupt (\cite{segmentation-corruption}). On the other hand, private schools have a richer clientele, with teachers that are paid less and are easier to change \cite{private-public-salaries}. Furthermore, private school teachers do not necessarily have a teaching degree - only "extensive experience working in a particular industry" when available, or, completely unqualified teachers are hired to fill the numbers (\cite{private-public-salaries}). Additionally, these teachers typically have smaller classes and consequently, a closer relationship with students, meaning they will have more time to pursue career development and are stronger role models for their students (\cite{private-public-salaries}). 
            
        \subsubsection{Targeting}
            
            It should be noted that there are two groups that need to be targetted: resource generators, and resource users. Luckily, enough content has been generated and thus no further efforts are needed in targeting them (H. Subair, personal communication, March 23, 2021). In regards to resource users, there are some considerations to be made. Contacting each potential client, that is, each teacher, individually is a very expensive task. Instead, the head of their organization, or an individual with many such teachers on their staff is the ideal target. Thus, private school owners are the ideal market for 1MT, as they can share the resource with their staff.

        \subsubsection{Customer Persona: Kwame "I want my students to have a better life" Zane}
            \begin{multicols}{2}
                \includegraphics[width=4.5cm, height=5cm]{man} \\
                \begin{large}
                    About Kwame:
                \end{large}
                \begin{itemize}
                    \item Age: 45 years old
                    \item Education: University Graduate
                    \item Occupation: Private school owner, 24 years
                    \item Relationship Status: Married
                    \item Family Status: Lives with his wife and 4 children
                \end{itemize}
            \end{multicols}
            \begin{multicols}{2}
                \begin{large}
                    Behaviour
                \end{large}
                \begin{itemize}
                    \item Tries to make teachers feel like they are involved and have a say
                    \item Tries to ensure that every student who goes through his school has their life changed for the better
                    \item Speaks to his teachers monthly regarding any issues, but not many items are brought up
                \end{itemize}
                \begin{large}
                    Needs, Wants and Expectations
                \end{large}
                \begin{itemize}
                    \item Needs empowered teachers that will increase the school's reputation and change their lives for the better
                    \item Wants to prevent teacher turnover and have replacements for when they retire
                    \item Expects to keep running the school for the next 2 decades
                \end{itemize}
            \end{multicols}
            Kwame has about 30 teachers in his staff, all of whom could use empowerment. At first, he was sceptical about 1MT and is hoping that a trial run will prove worth it.
    
    \subsection{Company Offerings}
        
        1MT offers a series of online modules to empower teachers. Their black belt program is currently their most successful and it is the hope that said program need not be modified for different users (H. Subair, personal communication, March 23, 2021, \cite{companysite-programs}). 1MT also offers programs targetted to specific demographics (ie women), exchanges and an SDG. Pricing is negotiable, depending on the client's circumstances and currently, private school owners are being targeted (H. Subair, personal communication, March 23, 2021).

\section{Analysis of Venture}

        % \subsubsection{Value Proposition}
        %     \subsubsubsection{Statement}
        %     \subsubsubsection{Canvas}
        % \subsubsection{Positioning}
        %     \subsubsubsection{Statement}
        %     \subsubsubsection{Map}


    % \section{Go to Market Execution}
    % \subsection{Product} \label{Product}
    %     Product description: key attributes, level of customization, unique value
    %     Brand development strategy and related branding decisions (e.g. name, logo, color, etc.)
    % \subsection{Price}
    %     Pricing approach and reasoning for it
    %     Product margins
    %     Pricing strategy used and why: cost based, value based, penetration or skimming
    %     Pricing adjustments: e.g., sales promotions, segmented pricing, etc.
    % \subsection{Distribution}
    %     Type(s) of channel used and why
    %     Distribution strategy adopted and why
    %     Where the product is available/number of outlets and trade promotions (if any)
    %     Potential channel conflicts (if appropriate) and mitigation plan
    % \subsection{Promotion}
    %     Communication plan (i.e. SMART communication objectives, key message, target segment, budget, etc.)
    %     Develop an appropriate integrated marketing communication (IMC) plan. Be specific and explain your reasons for choosing each communication element and how to create synergies between the different IMC elements chosen. It is important to consider the costs and viability of all elements of your IMC plan. For example, for advertising, remember to include details of the message strategy, execution style, and the media strategy (e.g. media choice, media scheduling pattern, etc.)

    \subsection{Company Analysis}
        
        \begin{figure}[h!]
            \begin{center}
                \begin{tabular}{| p{7.5cm} | p{7.5cm} |}
                    \hline
                    \\[-1em]
                    \begin{large}\textbf{Strength}\end{large} & \begin{large}\textbf{Weakness}\end{large}\\
                    \hline
                    Powerful mission - 1MT will be able to draw in sponsors through its promotion &
                        Limited brand awareness - 1MT has little "proof" as being useful and uncorrupt \\
                    Infrastructure set up in Sub-Saharan Africa - 1MT will be able to operate in Sub-Saharan [technological] conditions (ie inconsistent internet) & 
                        Potential customers may not have the funds \\
                    There are few, if any, for-profit competitors in the Sub-Saharan market &
                        Customers may try to claim not having funds to get a better deal due to open price negotiation \\
                    1MT is trying to rewrite a pervasive social stigma &
                        1MT is trying to rewrite a pervasive social stigma \\
                    \hline
                    \\[-1em]
                    \begin{large}\textbf{Opportunity}\end{large} & \begin{large}\textbf{Threat}\end{large}\\
                    \hline
                    Continue solely for social good instead of social good and profit - will allow partnerships to other related organizations(Khan Academy) &
                        Other cheaper or free E-learning tools / organizations - may prove more attractive for users \\
                    Partner with other companies such that they pay 1MT to have their content shared through 1MT's network &
                        Should 1MT make a profit, other more well known brands will seek to penetrate the market \\
                    \hline
                \end{tabular} \\
                \caption{Table depicting the strengths weaknesses opportunities and threats to 1MT}
            \end{center}
        \end{figure}

    \subsection{Competitive Forces}

        The competitive forces discussed will be that of powerful buyer groups, powerful supplier groups, competitive rivalry, the threat of new entrants and the threat of substitution. At the moment, success is quantified by the number of customers. Furthermore, 1MT is an e-learning tool. As such, the buyer groups are very powerful - 1MT does not currently have enough connections to ignore any internationally recognized they do have. To combat this, use the existing prized buyers to cement the brand's authenticity, and use the new authority to attract more customers. There is only one powerful supplier group, that is the content created by subject matter experts. However, there is a surplus of raw content created and thus, they are not a factor. The threat of substitution, new entrants and the competitive rivalry within the market are all things of note. All three exist as powerful forces within the industry but outside of the Sub-Sahara African region. The moment things appear profitable, they will seek to exploit this market as well. To combat this, there are two potential courses of action. First, secure the branding within the area so that if others try to exploit the market, they will not be able to. Alternately, 1MT can make use of the old adage "If you can't beat them, join them," and partner with potential substitutes, rivals etc. 1MT will take resources from their new partners and propagate it through their network.

\section{Recommendations}

    It is recommended that the executive team change 1MT into a social good enterprise. 1MT's greatest strength is that of its mission, and the fact that it has infrastructure in place. 1MT is neither an unique product, not the cheapest. It also does not have much authenticity in the eyes of the public compared to existing brands, and there is no proof that 1MT's product is better than these organizations. E-learning platforms are most attractive when they are either extremely specialized (it gives an appearance of authority) or a centralized platform for many topics. For-Profit competitors have not entered the Sub-Saharan market likely because it was not deemed profitable. Furthermore, once this change is made, 1MT will be able to partner with other like organizations. Then they will be able to focus on combatting the stigma by using their infrastructure to share resources to their clients, as opposed to creating new resources, when there are existing ones that are tried and proven. It is proven that for the most part in Sub-Saharan Africa, students who do better are those who's parents can afford private school (\cite{private-public-success}). These students typically move on to "better" professions, but they are also of the smaller population. To better address the social issue, the public schools must also go through this change. It will take a very long time for even the smallest iota of success, but the possibility is there.

    Should 1MT continue with being a for-profit organization, it is recommended that success be quantified by some monetary measure rather than customers reached (H. Subair, personal communication, March 23, 2021). It should no longer allow price negotiation for customers. 
    
    For both scenarios, 1MT should [attempt to] partner with internationally recognized organizations to build their brand. As 1MT has enough raw resources, both a marketing team and editor should be hired to ensure customers can be reached, and to create more usable material. 1MT should not focus on expanding to other subjects, or creating new programs, but rather, enhancing existing programs. Furthermore, the marketing team (or person) hired should work on developping a promotional campaign to prospective users. User success stories should continue to be used, and information on programs easier to access. For example, the document available for download at https://1millionteachers.com/programs-v/the-black-belt-program/ should be made mroe professional and contain legitimate program details as opposed to desired personality traits.

\section{Conclusion}

    The plan proposed by 1 Million Teachers' co-founder is unlikely to be profitable. The aim to empower teachers is admirable, but once teachers are "empowered," this tool will be obsolete. Further, tackling the stigma (of teaching being a lowly profession) by empowering teachers is something that will take somewhere near a decade to pass - teachers must get trained, they must perfect any new skills they learn and hope students will desire to become like them, and teacher's college usually takes around five years. In a decade, there is no guarantee that 1MT will still be the only or the best tool in the field, and if 1MT was instrumental to the deconstruction of the stigma, and the company went bankrupt, said stigma would remain. On the other hand, 1MT has a higher chance of success if it changes into a social-good enterprise. 1MT will have advice and resources from known organizations with experience in the industry and be better for it.


\medskip

\cleardoublepage
\printbibliography

\end{document}